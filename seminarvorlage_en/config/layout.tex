%===============================================================================
% Zweck: KTR-Seminar-Vorlage in Anlehung an G. Wirtz, Lehrstuhl Praktische Informatik
%===============================================================================
%===============================================================================
% zentrale Layout-Angaben und Befehle
%===============================================================================
%
% \usepackage{german}
\usepackage[english]{babel}
%\usepackage[latin1]{inputenc}
\usepackage{fancyhdr}
\usepackage[T1]{fontenc}
\usepackage{ae}
\usepackage{color}
\usepackage{amsmath}
\usepackage{amsfonts}
%
\ifpdf
\usepackage[pdftex,bookmarksopen,bookmarksnumbered,pdfborder=0]{hyperref}
\usepackage[pdftex]{graphicx}
\pdfcompresslevel=9
\else
\usepackage{url}
\usepackage[dvips]{graphicx}
\fi
%\usepackage[dvips]{rotating}
%
% ausf\"{u}hrlichere Fehlermeldungen
\errorcontextlines=999
%
% Page-Layout: A4 aus Header
% Alternative
%\setlength\headheight{14pt}
%\setlength\topmargin{-15,4mm}
%\setlength\oddsidemargin{-0,4mm}
%\setlength\evensidemargin{-0,4mm}
%\setlength\textwidth{160mm}
%\setlength\textheight{252mm}
%
% Absatzeinstellungen
\setlength\parindent{0mm}
\setlength\parskip{2ex}
%
% Anweisung zur Erstellung der Titelseite
% #1 = Name des Seminars
% #2 = Name der Ausarbeitung
% #3 = Autor
% #4 = Betreuer
% #5 = Semester, z.B. WS 2007/08
\renewcommand{\maketitle}[5] {
  \begin{titlepage}
  \centering
    \begin{minipage}[t]{16cm}
      \hfill
      \begin{minipage}{12cm}
        \centering
        Otto-Friedrich-University of Bamberg
        \\[12pt]
        {\Large Professorship for Computer Science,
        \\
        Communication Services, Telecommunication Systems and Computer Networks}
      \end{minipage}
      \hfill
      \begin{minipage}{3cm}
        \includegraphics[height=28mm]{config/images/logo} %height=26mm
      \end{minipage}
    \end{minipage}\\[70pt]%[50pt]
   % 
    {\Large\bf Seminar on}
    \\[36pt]
   % oder: {\LARGE Ausarbeitung des KTR-Seminars}\\[12pt]
    {\LARGE #1}\\[80pt]
    {\Large\bf Topic:}\\[36pt]
    {\LARGE\bf #2}\\
    \vfill
    \begin{minipage}{\textwidth}
      \center
      Submitted by:\\
      {\Large #3\\[18pt]}
      Supervisor: #4\\[12pt]
      Bamberg, #5
    \end{minipage}
  \end{titlepage}
}
%
% Erstellung von Abk\"{u}rzungsverzeichnis
\newcommand{\abbrev}[2]{#1 & #2\\}
\newcommand{\abkuerzungen}{
\section*{Abbreviations}
\hspace{2ex}
\begin{tabular}{ll}
\input{abkuerzungen.tex}
\end{tabular}
}
%
% Einbindung eines Bildes mit angegebener Breite
% #1 = label f\"{u}r \ref-Verweise
% #2 = Name des Bildes ohne Endung relativ zu Bilder-Verzeichnis
% #3 = Beschriftung
% #4 = Breite des Bildes im Dokument in cm
\newcommand{\bildw}[4]{%
  \begin{figure}[htb]%
    \centering
    \includegraphics[width=#4cm]{Bilder/#2}%
    \vskip -0.3cm%
    \caption{#3}%
    \vskip -0,2cm%
    \label{#1}%
  \end{figure}%
}
%
% Einbindung eines Bildes mit Seitenbreite
% #1 = label f\"{u}r \ref-Verweise
% #2 = Name des Bildes ohne Endung relativ zu Bilder-Verzeichnis
% #3 = Beschriftung
\newcommand{\bild}[3]{%
  \begin{figure}[htb]%
    \centering%
    \includegraphics[width=\textwidth]{Bilder/#2}%
    \vskip -0.3cm%
    \caption{#3}%
    \vskip -0,2cm%
    \label{#1}%
  \end{figure}%
}
%
\numberwithin{equation}{section}
%
%===============================================================================
% zentrale Layout-Angaben und Befehle
%===============================================================================
%
