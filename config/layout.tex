%===============================================================================
% Zweck: KTR-Seminar-Vorlage in Anlehung an G. Wirtz, Lehrstuhl Praktische Informatik
%===============================================================================
%===============================================================================
% zentrale Layout-Angaben und Befehle
%===============================================================================
%
% \usepackage{german}
\usepackage{ifthen}
% Language Selection for Babel
\ifthenelse{\equal{\lang}{ngerman}}{\usepackage[german, \lang]{babel}}{\usepackage[\lang]{babel}}

\usepackage[utf8]{inputenc}
\usepackage{fancyhdr}
\usepackage[T1]{fontenc}
\usepackage{ae}
\usepackage{color}
\usepackage{amsmath}
\usepackage{amsfonts}
%

\usepackage[pdftitle={\@title},pdfauthor={\@author},pdftex,bookmarksopen,bookmarksnumbered,pdfborder=0]{hyperref}
\usepackage[pdftex]{graphicx}
\pdfcompresslevel=9

%\usepackage[dvips]{rotating}
%
% ausf\"{u}hrlichere Fehlermeldungen
\errorcontextlines=999
%
% Page-Layout: A4 aus Header
% Alternative
\setlength\headheight{14pt}
\setlength\topmargin{-15,4mm}
\setlength\oddsidemargin{-0,4mm}
\setlength\evensidemargin{-0,4mm}
\setlength\textwidth{160mm}
\setlength\textheight{252mm}
%
%% Absatzeinstellungen
\setlength\parindent{0mm}
\setlength\parskip{2ex}


\makeatletter
\renewcommand{\maketitle} {
  \begin{titlepage}
  \centering
    \begin{minipage}[t]{16cm}
      \hfill
      \begin{minipage}{12cm}
      \ifthenelse{\equal{\lang}{ngerman}}{%
        \centering
        Otto-Friedrich-Universit\"at Bamberg
        \\[12pt]%
        {\Large Professur f\"ur Informatik\\[.5em]%
        \large insbesondere Kommunikationsdienste,\\[.5em]%
        Telekommunikationsdienste und Rechnernetze}%
        }{%
        \centering
        Otto-Friedrich-University of Bamberg
        \\[12pt]
        {\Large Professorship for Computer Science,\\[.5em]%
        \large Communication Services, Telecommunication\\[.5em]%
        Systems and Computer Networks}%
        }
      \end{minipage}
      \hfill
      \begin{minipage}{3cm}
        \includegraphics[height=28mm]{config/images/logo} %height=26mm
      \end{minipage}
    \end{minipage}\\[70pt]%[50pt]
   % 
    {\Large\bf \ifthenelse{\equal{\lang}{ngerman}}{Ausarbeitung des KTR-Seminars}{Seminar on}}
    \\[36pt]
   % oder: {\LARGE Ausarbeitung des KTR-Seminars}\\[12pt]
    {\LARGE \@title}\\[80pt]
    {\Large\bf \ifthenelse{\equal{\lang}{ngerman}}{Thema:}{Topic:}}\\[36pt]
    {\LARGE\bf \subtitle}\\
    \vfill
    \begin{minipage}{\textwidth}
      \center
      \ifthenelse{\equal{\lang}{ngerman}}{Vorgelegt von:}{Submitted by:}\\
      {\Large \@author \\[18pt]}
      \ifthenelse{\equal{\lang}{ngerman}}{Betreuer:}{Supervisor:} \supervisor \\[12pt]
      Bamberg, \@date
    \end{minipage}
  \end{titlepage}
}
\makeatother
%
\usepackage[intoc]{nomencl}
\ifthenelse{\equal{\lang}{ngerman}}{%
\renewcommand{\nomname}{Abkürzungsverzeichnis}}%
{\renewcommand{\nomname}{List of Abbreviations}}
\renewcommand{\nomlabel}[1]{#1 \dotfill}
\makenomenclature

% Einbindung eines Bildes mit angegebener Breite
% #1 = label f\"{u}r \ref-Verweise
% #2 = Name des Bildes ohne Endung relativ zu Bilder-Verzeichnis
% #3 = Beschriftung
% #4 = Breite des Bildes im Dokument in cm
\newcommand{\bildw}[4]{%
  \begin{figure}[htb]%
    \centering
    \includegraphics[width=#4cm]{Bilder/#2}%
    \vskip -0.3cm%
    \caption{#3}%
    \vskip -0,2cm%
    \label{#1}%
  \end{figure}%
}
%
% Einbindung eines Bildes mit Seitenbreite
% #1 = label f\"{u}r \ref-Verweise
% #2 = Name des Bildes ohne Endung relativ zu Bilder-Verzeichnis
% #3 = Beschriftung
\newcommand{\bild}[3]{%
  \begin{figure}[htb]%
    \centering%
    \includegraphics[width=\textwidth]{Bilder/#2}%
    \vskip -0.3cm%
    \caption{#3}%
    \vskip -0,2cm%
    \label{#1}%
  \end{figure}%
}
%
\numberwithin{equation}{section}
%
%===============================================================================
% zentrale Layout-Angaben und Befehle
%===============================================================================
%
