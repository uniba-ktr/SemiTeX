%===============================================================================
% Zweck:    KTR-Seminar-Vorlage
% Erstellt: 16.10.2007
% Updated:  27.06.2016
% Autor:    U.K. / M.G.
%===============================================================================
\RequirePackage[hyphens]{url}
\documentclass[journal, onecolumn, a4paper, 12pt]{IEEEtran}
%===============================================================================
% zentrale Layout-Angaben und Befehle
%===============================================================================
\newcommand\meta{./meta}

%===============================================================================
% Zweck: KTR-Seminar-Vorlage in Anlehung an G. Wirtz, Lehrstuhl Praktische Informatik
%===============================================================================
%===============================================================================
% zentrale Layout-Angaben und Befehle
%===============================================================================

% Language Selection for Babel
\input{config/language}
% What You should change:
% Here goes your name
\author{Author}
% and the title of your seminar
\newcommand{\subtitle}{Your Topic on the Seminar}
% the date of the submission
\date{\today}

% What is already done

\newlanguagecommand{\semester}
\addtolanguagecommand{\semester}{english}{Winter Term 2016/17}
\addtolanguagecommand{\semester}{ngerman}{Wintersemester 2016/17}

\newlanguagecommand{\ltitle}
\addtolanguagecommand{\ltitle}{english}{Content-Centric Software-Defined Networking (SDN) and Mobile Edge Computing}
\addtolanguagecommand{\ltitle}{ngerman}{Content-Centric Software-Defined Networking (SDN) and Mobile Edge Computing}

\title{\ltitle}
\newcommand{\supervisor}{Prof. Dr. Udo Krieger}

\gittrue
\seminartrue


\usepackage[utf8]{inputenc}
\usepackage{fancyhdr}
\usepackage[T1]{fontenc}
\ifgit
  \usepackage[mark]{gitinfo2}
\fi
\usepackage{ae}
\usepackage{color}
\usepackage{amsmath}
\usepackage{amsfonts}

%%   Fuer anspruchsvolle Tabellen   %%
\usepackage{longtable, colortbl}
\usepackage{multicol, multirow}

\usepackage[pdftitle={\@title},pdfauthor={\@author},pdftex,bookmarksopen,bookmarksnumbered]{hyperref}
\usepackage[pdftex]{graphicx}
\usepackage{float}
\usepackage{tikz}
\usepackage{pgfplots}
\usetikzlibrary{arrows,shapes,fit,positioning,decorations,backgrounds,shadows}

\pdfcompresslevel=9

% Code-Hervorhebung
% Quellcode
\usepackage{verbatim}            % Quellcode einbinden (\verbatiminput) standardpaket
\usepackage{moreverb}
% PseudoCode
\usepackage{algorithm}
\usepackage{algpseudocode}
%\usepackage{algorithmicx}

\floatname{algorithm}{\algo}
\algrenewcommand{\algorithmiccomment}[1]{\hskip1em\textcolor{gray!60}{$\rhd$ #1}}
\renewcommand{\listalgorithmname}{\loa}
\def\algorithmautorefname{\algo}


%%   intoc zur Aufnhame des Abkuerzungs- und Symbolverzeichnisses ins Inhaltsverzeichnis
\usepackage[intoc]{nomencl}
\setlength{\nomlabelwidth}{.20\hsize}
\renewcommand{\nomlabel}[1]{#1 \dotfill}
\setlength{\nomitemsep}{-\parsep}
\makenomenclature

\renewcommand{\nomname}{\abbr}

%%   Hervorhebung der Abkuerzungsbuchstaben   %%
\usepackage[normalem]{ulem}
\newcommand{\m}[1]{\uline{#1}}

% ausf\"{u}hrlichere Fehlermeldungen
\errorcontextlines=999
%
% Page-Layout: A4 aus Header
% Alternative
\setlength\headheight{14pt}
\setlength\topmargin{-15,4mm}
\setlength\oddsidemargin{-0,4mm}
\setlength\evensidemargin{-0,4mm}
\setlength\textwidth{160mm}
\setlength\textheight{252mm}
%
%% Absatzeinstellungen
\setlength\parindent{0mm}
\setlength\parskip{2ex}



\makeatletter
\renewcommand{\maketitle} {
  \begin{titlepage}
  \centering
    \begin{minipage}[t]{16cm}
      \hfill
      \begin{minipage}{12cm}
            \centering
        \uni
        \\[12pt]%
        {\Large \chair\\[.5em]%
        \large \chairsub}%
      \end{minipage}
      \hfill
      \begin{minipage}{3cm}
        \includegraphics[height=28mm]{config/images/logo} %height=26mm
      \end{minipage}
    \end{minipage}\\[70pt]%[50pt]
    {\Large\bf \seminar}
    \\[36pt]
    {\LARGE \@title}\\[80pt]
    {\Large\bf \topic:}\\[36pt]
    {\LARGE\bf \subtitle}\\
    \vfill
    \begin{minipage}{\textwidth}
      \center
      \submitter:\\
      {\Large \@author \\[18pt]}
      \lsupervisor: \supervisor \\[12pt]
      Bamberg, \@date\\
      \semester
    \end{minipage}
  \end{titlepage}
}
\makeatother
%

% Einbindung eines Bildes mit angegebener Breite
% #1 = label f\"{u}r \ref-Verweise
% #2 = Name des Bildes ohne Endung relativ zu Bilder-Verzeichnis
% #3 = Beschriftung
% #4 = Breite des Bildes im Dokument in cm
\newcommand{\bildw}[4]{%
  \begin{figure}[htb]%
    \centering
    \includegraphics[width=#4cm]{Bilder/#2}%
    \vskip -0.3cm%
    \caption{#3}%
    \vskip -0,2cm%
    \label{#1}%
  \end{figure}%
}
%
% Einbindung eines Bildes mit Seitenbreite
% #1 = label f\"{u}r \ref-Verweise
% #2 = Name des Bildes ohne Endung relativ zu Bilder-Verzeichnis
% #3 = Beschriftung
\newcommand{\bild}[3]{%
  \begin{figure}[htb]%
    \centering%
    \includegraphics[width=\textwidth]{Bilder/#2}%
    \vskip -0.3cm%
    \caption{#3}%
    \vskip -0,2cm%
    \label{#1}%
  \end{figure}%
}
%
\numberwithin{equation}{section}
%
%===============================================================================
% zentrale Layout-Angaben und Befehle
%===============================================================================
%
%#1 Breite
%#2 Datei (liegt im image Verzeichnis)
%#3 Beschriftung
%#4 Label fuer Referenzierung
\newcommand{\image}[4]{%
\begin{figure}[H]%
\centering%
\includegraphics[width=#1]{image/#2}%
\caption{#3}%
\label{#4}%
\end{figure}%
}

%#1 Datei (liegt im graphic Verzeichnis)
%#2 Beschriftung
%#3 Label fuer Referenzierung
%#4 Skalierungsfaktor
\newcommand{\scaletikzimage}[4]{%
\begin{figure}[H]%
\centering%
\scalebox{#4}{%
\input{graphic/#1.tikz}}%
\caption{#2}%
\label{#3}%
\end{figure}
}

%#1 algorithm name
%#2 algorithm label
%#3 file name in code-folder
\newcommand{\pseudo}[3]{%
\small%
\begin{algorithm}[H]%
\caption{#1}%
\label{#2}%
\input{code/#3.tex}%
\end{algorithm}%
\normalsize%
}


%===============================================================================
% LATEX-Dokument
%===============================================================================
\input{\meta/config/hyphenation}
\begin{document}
  %===============================================================================
  % Zum Kompilieren pdflatex und bibtex ausführen.
  % Konfiguration in texmaker: Options -> Configure Texmaker -> Quick Build -> Select Latexmk + ViewPDF
  % Entsprechende Informationen in den config/metainfo verändern
  % Zur Auswahl der Sprache im folgenden Befehl
  % ngerman für deutsch eintragen, english für Englisch.git@github.com:uniba-ktr/SemiTeX.git
  %===============================================================================
\selectlanguage{english}

\maketitle

\pagenumbering{Roman}
\setcounter{page}{2}
%
\tableofcontents
% Einstellungen f\"{u}r Literaturverzeichnis
\newpage
\addcontentsline{toc}{section}{\listfigurename}
\listoffigures
\newpage
\addcontentsline{toc}{section}{\listtablename}
\listoftables
\newpage
\printnomenclature
%===============================================================================
% LATEX-Dokument: Kapitel laden
%===============================================================================
%
\newpage
\pagenumbering{arabic}
\setcounter{page}{1}
%
% to use git tagging
%
\ifgit
      \section{VERSION}
    \subsection{git}
    \#: \gitAbbrevHash\\
    @: \gitAuthorIsoDate\\
    \gitReferences
    \subsection{gitinfo2 -- setup}
    \href{https://www.ctan.org/tex-archive/macros/latex/contrib/gitinfo2}{Gitinfo 2}
    \subsubsection{git hooks}
    To fill watermark at buttom, deploy gitinfo2-hook.txt to githooks: (copy and make executable) or use \texttt{make git}
    \begin{itemize}
        \item .git/hooks/post-checkout
        \item .git/hooks/post-commit
        \item .git/hooks/post-merge
    \end{itemize}
    \subsubsection{remove watermark}
    To disable watermark, remove option \texttt{[mark]} from \textbackslash usepackage[mark]\{gitinfo2\} in \textit{config/commands.tex} at line 16.

\fi
%
% hier einzelne Kapitel mit \input{Kapitel-File} einf\"{u}gen
%
\section{Einleitung}\label{sec:ein}
Einleitung nach \autoref{sec:ein}

\section{Hauptteil}\label{sec:haupt}
\subsection{Bilder und Grafiken}\label{subsec:grafiken}
\subsubsection{Bilder}\label{subsubsec:bilder}
Bilder befinden sich im Image-Orgner und lassen sich mit \textbackslash image\{Breite\}\{Datei im Image-Verzeichnis\}\{Beschriftung\}\{Label\} einbinden. \image{3cm}{logo.png}{Uni-Logo}{img:uni} Die Referenzierung erfolg mittels \textbackslash autoref\{Label\}, also z.B. \autoref{img:uni}.
\subsubsection{Grafiken mit TikZ}
Grafiken im TikZ-Framework\footnote{\url{http://www.tn-home.de/TUGDD/Stuff/TikZ_final.pdf}} lassen sich mit dem Befehl \textbackslash scaletikzimage\{Datei im Image Verzeichnis\}\{Beschriftung\}\{Label\}\{Skalierungsfaktor\} einbinden. \scaletikzimage{tikz}{TikZ-Grafik}{img:tikz}{0.9}
\subsection{Tabellen}
Tabellen lassen sich mit dem Environment\\
\textbackslash begin\{longtable\}[H h t b c]\{Spaltendefinitionen\} ...\\
\qquad\qquad \textbackslash caption\{Tabellenunterschrift\}\\
\qquad\qquad \textbackslash label\{Label\}\\
\textbackslash end\{longtable\}\\
 definieren\footnote{\url{ftp://ftp.dante.de/pub/tex/macros/latex/required/tools/longtable.pdf}}\\
\begin{longtable}[H]{|p{0.2\textwidth}|p{0.2\textwidth}|p{0.2\textwidth}|}
\hline
A&B&C\\
\hline
\caption{Tabelle 1}
\label{tab:tab1}
\end{longtable}
\subsection{Code-Ausschnitte}
Pseudo-Code Ausschnitte lassen sich mit \textbackslash pseudo\{Name des Algorithmus\}\{Label\}\{Datei im Code-Verzeichnis\} einbinden.
\pseudo{Mittelwert}{lst:mean}{code}
\section{Zitate}
Mit \textbackslash nocite*\{\} lassen sich alle Einträge in der Bibliography ausgeben. Mit \textbackslash cite[S. xx]\{Key\} lassen sich Zitate einfügen. Z.B. \cite[S. 234]{Kurose12} \nocite*{}
\subsection{Abkürzungen}

Abkürzungen können mit \textbackslash nomenclature\{Abk\}\{\textbackslash m\{Abk\}ürzung\} \nomenclature{Abk}{\m{Abk}ürzung} angegeben werden. Diese werden alphabetisch sortiert in ein Abkürzungsverzeichnis aufgenommen.


%
%===============================================================================
% LATEX-Dokument: Literaturverzeichnis
%===============================================================================
%
\newpage
\phantomsection
% Einstellungen f\"{u}r Literaturverzeichnis
\addcontentsline{toc}{section}{\bibname}

\bibliographystyle{IEEEtran}
% argument is your BibTeX string definitions and bibliography database(s)
\bibliography{\meta/exampleLiterature/bib}
% Nutzung von Bibtex:
% hier den bib-file einbinden
%
% GATHER{bibfile.bib}
% \footnotesize
% \bibliography{bibfile}
% ansonsten: bbl als tex Datei einbinden
 %\input{KTR-Seminar-Literatur.tex}
%===============================================================================
% LATEX-Dokument: Literaturverzeichnis
%===============================================================================
\erklaerung
\end{document}
